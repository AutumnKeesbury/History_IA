% !TeX root = ../HistoryIA.tex

\documentclass{subfiles}
\begin{document}

This paper seeks to investigate the question: \flq{} \textit{How did the Import of Raw Cotton to Great Britain Influence Its Willingness to Recognize the
Confederate States of America?}\frq{}. It is well-known that the southern United States is a largely agrarian region of the nation, and further that the 
economy of these states has, since the colonial era, been heavily dependent on the exportation of various \flq{}cash crops\frq{}. Cotton, the crop with which 
the 19th century south is most associated, became the staple export of the region, and indeed may be argued to have kick-started or at least supported the 
lucrative textile industry of Great Britain. In this paper I examine the economic codependency of the Antebellum South (and subsequently the of the 
Confederate States of America [CSA]) with Great Britain, and the effects of this relationship on the prospect of the British Parliament's diplomatic 
recognition of the CSA as a sovereign nation. To do this, I shall rely on the vast contemporary records of exports and manufactures, the body of post-
contemporary literature regarding the relations of both nations, and the large base of correspondences and personal records of those involved.

A source that shall be relied upon heavily in assessing the economic state of the Antebellum South in the decades leading up to the war is 
\say{\Citetitle{u.s.congressImportsduties1884}} from the first session of the 48th Congress of the United States of America, held in 1884. The purpose of
this source 

\end{document}