% !TeX root = ../HistoryIA.tex

\autocites(The data for figure \ref{fig.: B.1} comes from:)()[(Total Exports)][Table A-III]{northeconomicgrowth1966}[(Cotton Exports)][Part 2, Table 2]{u.s.congressImportsduties1884}
On the eve of the American Civil War, U.S. cotton accounted for over 88\% of the cotton imported into Great Britain \autocite[40]{baileyOtherSide1994}. This 
cotton was, as noted by Joseph Inikori, the key to the English economy in the mid-19th century: 
\begin{quote}
    Indeed, \say{the Industrial Revolution} in England, in the strict sense of the phrase, is little more than a revolution in [...] cotton textile production.
    \autocite[Joseph Inikori, {The Slave Trade and Revolution in Cotton Textile Production in England}, quoted in][40]{baileyOtherSide1994}
\end{quote}
The importance of this economic relationship for the Confederate cause was not lost on the rebelling sates, and unquestionably played a significant role in their 
foreign policy towards England throughout the course of the war. However, the extent to which this was an effective diplomatic strategy is not comparably evident,
and is the central focus of this investigation. 

\begin{wrapfigure}{R}{0.5\textwidth}\label{fig.: B.1}
    \begin{tikzpicture}
\begin{axis}[
	grid = both,
	major grid style = {lightgray},
	minor grid style = {lightgray!50},
	minor tick num = 1,
	xticklabel style={
		/pgf/number format/1000 sep=,
		rotate=0,anchor=north,
		font=\scriptsize
	},
	y tick label style={
		/pgf/number format/.cd,
		fixed,
		fixed zerofill,
		precision=0,
		/tikz/.cd,
		font=\scriptsize
	},
	ymin=0,
	ymax= 300000000,
	scaled y ticks=manual:{}{\pgfmathparse{#1 / 10^7}},
]

\addplot[thick, black!80!white] table [x = {Year}, y = {Total_Cotton_Export}] {DATA/cottonexps.dat};

\addplot[thick, dashed] table [x ={Year}, y = {Raw_Cotton_Export}] {DATA/cottonexps.dat};

\end{axis}

\end{tikzpicture}
    \captionof{figure}{Value (hundreds of millions of dollars) of Total Exports 
                (solid) and of Total Raw Cotton Exports (dashed) in the U.S.: 1815-1860
    }
\end{wrapfigure}  

