% !TeX root = ../HistoryIA.tex


\Autocites(Data in Figure~\ref{fig: B.i}:)()[Total exports:][Part 2, Table 2]{u.s.congressImportsduties1884}[Cotton exports:][Table A-III]{northeconomicgrowth1966}
On the eve of the American Civil War, U.S. cotton accounted for over 88\% of the cotton imported into Great Britain \autocite[40]{baileyOtherSide1994}. This 
cotton was, as noted by Joseph Inikori, the key to the English economy in the mid-19th century: 
\begin{quote}
    Indeed, \say{the Industrial Revolution} in England, in the strict sense of the phrase, is little more than a revolution in [...] cotton textile production.
    \autocite[Joseph Inikori, {The Slave Trade and Revolution in Cotton Textile Production in England}, quoted in][40]{baileyOtherSide1994}
\end{quote}
On the other side of the Atlantic, in the continental United States, in the time leading up to the war, the amount and value of exported goods skyrocketed, 
and, though by 1860 the percentage of the total value of exports that cotton represented was lower than it had been in 1851, it was at an all-time high in 
terms of dollar-value (Figure~\ref{fig: B.i}). The importance of this economic relationship for the Confederate cause was not lost on the rebelling sates, 
and unquestionably played a significant role in their foreign policy towards England throughout the course of the war. However, the extent to which this was an 
effective diplomatic strategy is not comparably evident, and is the central focus of this investigation. 

\begin{figure}[ht]
    \centering
    \begin{tikzpicture}
\begin{axis}[
	grid = both,
	major grid style = {lightgray},
	minor grid style = {lightgray!50},
	minor tick num = 1,
	xticklabel style={
		/pgf/number format/1000 sep=,
		rotate=0,anchor=north,
		font=\scriptsize
	},
	y tick label style={
		/pgf/number format/.cd,
		fixed,
		fixed zerofill,
		precision=0,
		/tikz/.cd,
		font=\scriptsize
	},
	ymin=0,
	ymax= 300000000,
	scaled y ticks=manual:{}{\pgfmathparse{#1 / 10^7}},
]

\addplot[thick, black!80!white] table [x = {Year}, y = {Total_Cotton_Export}] {DATA/cottonexps.dat};

\addplot[thick, dashed] table [x ={Year}, y = {Raw_Cotton_Export}] {DATA/cottonexps.dat};

\end{axis}

\end{tikzpicture}
    \caption{
        Total value of exports from the United States (\textemdash) 
        and total value of raw cotton exports (- -);
        Percentage of total value of U.S. exports coming from cotton (\textcolor{red!75!black}{\textemdash}).
    }
    \label{fig: B.i}
\end{figure} 

In terms of end goals, the CSA had two when it came to diplomacy with the English: (1) an official recognition of the sovereignty of the nation, with the
understanding that this would lead to mediation of the conflict by Britain, and (2) the removal of the Union naval blockade on Southern ports
\Autocites()()[][pp. 383]{mcphersonBattleCry1988}[W.L. Yancey and A. Dudley Mann to Secretary of State R. Toombs, July 15, 1861, in:][pp. 45]{davismessagespapers1966}.
It is the latter issue's nature terms of international law, and effects on the cotton exports of the south, that would play the most significant role in
the Anglo-Confederate relations during the early war, as it presented a goal which was \say{much more likely to be obtained within a reasonable time} than
recognition\autocite[John Slidell to Secretary of State R.M.T. Hunter, September 26, 1862, in:][pp. 187]{davismessagespapers1966}. Though this issue does not
directly relate of the answer of the question regarding the role of cotton in the recognition of the CSA as sovereign by the English, it does shed light on a 
crucial, yet heretofore unestablished, piece of this investigation: the importance of cotton in Anglo-Confederate diplomacy. 

Indeed, the main tactic employed by southern diplomats in attempting to secure British condemnation of the blockade was a stressing of the English dependence 
on the south's cotton exports. The 