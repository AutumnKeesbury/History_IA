% !TeX root = ../HistoryIA.tex

On the eve of the American Civil War, U.S. cotton accounted for over 88\% of the cotton imported into Great Britain \autocite[40]{baileyOtherSide1994}. This 
cotton was, as noted by Joseph Inikori, the key to the English economy in the mid-19th century: 
\begin{quote}
    Indeed, \say{the Industrial Revolution} in England, in the strict sense of the phrase, is little more than a revolution in [...] cotton textile production.
    \autocite[Joseph Inikori, {The Slave Trade and Revolution in Cotton Textile Production in England}, quoted in][40]{baileyOtherSide1994}
\end{quote}
The importance of this economic relationship for the Confederate cause was not lost on the rebelling sates, and unquestionably played a significant role in their 
foreign policy towards England throughout the course of the war. However, the extent to which this was an effective diplomatic strategy is not comparably evident,
and is the central focus of this investigation. 

In the continental United States, in the time leading up to the war, the amount and value of exported goods skyrocketed, and, though by 1860 the percentage of 
the total value of exports that cotton represented was lower than it had been in 1851, it was at an all-time high in terms of dollar-value (Figure~\ref{fig: B.i}).
Both the Union and Confederacy knew full well how important this reliance was to be regarding the future of the war's diplomatic front, and it was not long into the
war that the Union navy began its blockade on southern ports. In the early years of the conflict, between 1861 and 1862, this blockade served as the cornerstone of
Confederate foreign policy: 

\begin{figure}[ht]
    \centering
    \begin{tikzpicture}
\begin{axis}[
	grid = both,
	major grid style = {lightgray},
	minor grid style = {lightgray!50},
	minor tick num = 1,
	xticklabel style={
		/pgf/number format/1000 sep=,
		rotate=0,anchor=north,
		font=\scriptsize
	},
	y tick label style={
		/pgf/number format/.cd,
		fixed,
		fixed zerofill,
		precision=0,
		/tikz/.cd,
		font=\scriptsize
	},
	ymin=0,
	ymax= 300000000,
	scaled y ticks=manual:{}{\pgfmathparse{#1 / 10^7}},
]

\addplot[thick, black!80!white] table [x = {Year}, y = {Total_Cotton_Export}] {DATA/cottonexps.dat};

\addplot[thick, dashed] table [x ={Year}, y = {Raw_Cotton_Export}] {DATA/cottonexps.dat};

\end{axis}

\end{tikzpicture}
    \caption{
        Total value of exports from the United States\autocite[Table A-III]{northeconomicgrowth1966} (\textemdash) 
        and total value of raw cotton exports\autocite[Part 2, Table 2]{u.s.congressImportsduties1884} (- -);
        Percentage of total value of U.S. exports coming from cotton (\textcolor{red!75!black}{\textemdash}).
    }
    \label{fig: B.i}
\end{figure}