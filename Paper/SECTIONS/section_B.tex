% !TeX root = ../HistoryIA.tex


\Autocites(Data in Figure~\ref{fig: B.i}:)()[Total exports:][Part 2, Table 2]{u.s.congressImportsduties1884}[Cotton exports:][Table A-III]{northeconomicgrowth1966}
On the eve of the American Civil War, U.S. cotton accounted for over 88\% of the cotton imported into Great Britain \autocite[40]{baileyOtherSide1994}. This 
cotton was, as noted by Joseph Inikori, the key to the English economy in the mid-19th century: 
\begin{quote}
    Indeed, \say{the Industrial Revolution} in England, in the strict sense of the phrase, is little more than a revolution in [...] cotton textile production.
    \autocite[Joseph Inikori, {The Slave Trade and Revolution in Cotton Textile Production in England}, quoted in][40]{baileyOtherSide1994}
\end{quote}
On the other side of the Atlantic, in the continental United States, in the time leading up to the war, the amount and value of exported goods skyrocketed, 
and, though by 1860 the percentage of the total value of exports that cotton represented was lower than it had been in 1851, it was at an all-time high in 
terms of dollar-value (Figure~\ref{fig: B.i}). The importance of this economic relationship for the Confederate cause was not lost on the rebelling sates, 
and unquestionably played a significant role in their foreign policy towards England throughout the course of the war. However, the extent to which this was an 
effective diplomatic strategy is not comparably evident, and is the central focus of this investigation. 

\begin{figure}[ht]
    \centering
    \begin{tikzpicture}
\begin{axis}[
	grid = both,
	major grid style = {lightgray},
	minor grid style = {lightgray!50},
	minor tick num = 1,
	xticklabel style={
		/pgf/number format/1000 sep=,
		rotate=0,anchor=north,
		font=\scriptsize
	},
	y tick label style={
		/pgf/number format/.cd,
		fixed,
		fixed zerofill,
		precision=0,
		/tikz/.cd,
		font=\scriptsize
	},
	ymin=0,
	ymax= 300000000,
	scaled y ticks=manual:{}{\pgfmathparse{#1 / 10^7}},
]

\addplot[thick, black!80!white] table [x = {Year}, y = {Total_Cotton_Export}] {DATA/cottonexps.dat};

\addplot[thick, dashed] table [x ={Year}, y = {Raw_Cotton_Export}] {DATA/cottonexps.dat};

\end{axis}

\end{tikzpicture}
    \caption{
        Total value of exports from the United States (\textemdash) 
        and total value of raw cotton exports (- -);
        Percentage of total value of U.S. exports coming from cotton (\textcolor{red!75!black}{\textemdash}).
    }
    \label{fig: B.i}
\end{figure} 

In terms of end goals, the CSA had two when it came to diplomacy with the English: (1) an official recognition of the sovereignty of the nation, with the
understanding that this would lead to mediation of the conflict by Britain, and (2) the removal of the Union naval blockade on Southern ports
\Autocites()()[][383]{mcphersonBattleCry1988}[W.L. Yancey and A. Dudley Mann to Secretary of State R. Toombs, July 15, 1861, in:][45]{davismessagespapers1966}.
It is the latter issue's nature terms of international law, and effects on the cotton exports of the south, that would play the most significant role in
the Anglo-Confederate relations during the early war, as it presented a goal which was \say{much more likely to be obtained within a reasonable time} than
recognition\autocite[John Slidell to Secretary of State R.M.T. Hunter, September 26, 1862, in:][187]{davismessagespapers1966}. Though this issue does not
directly relate of the answer of the question regarding the role of cotton in the recognition of the CSA as sovereign by the English, it does shed light on a 
crucial, yet heretofore non-established, piece of this investigation: the importance of cotton in Anglo-Confederate diplomacy. 

Indeed, the main tactic employed by southern diplomats in attempting to secure British condemnation of the blockade was a stressing of the English dependence 
on the south's cotton exports. This emphasis was reciprocated by some members of the Parliament, the most influential of whom was the Lord John Russell, although 
during the time of the blockade issue he did not yet firmly hold that an interventionist policy should be that of Whitehall: \say{It will not do for England and France to break a blockade for the sake of getting cotton}. \autocite[Russell to Palmerston, quoted in][Vol.I, p.199. There are two possible readings of what Russell said here, resting on one's interpretation of the phrase "It will not do". I have chosen to understand it, as I believe fits most reasonably within the context of the quote, as indicative of Russell's reservations regarding the breaking of the blockade, rather than a feeling that this action does not go far enough.]{adamsBritainAmericanWar1925}
The discourse regarding intervention in the blockade during these years developed a precedent for the relations of the two powers, made explicit by Prime Minister 
Palmerston in his response to Russell,
\begin{quote}
    [...] the want for cotton would not justify such a proceeding, unless, indeed, the distress created by that want was far more serious than it is likely to be.
    \Autocite[Vol.I, p.199]{adamsBritainAmericanWar1925}
\end{quote}

\subsection{The Lancashire Cotton Famine}
The Prime Minister's assertion of the relatively low want for cotton caused by the blockade would, in the late months of 1862, be challenged. Around this time,
what is referred to as the \flq Lancashire cotton famine\frq was beginning to take hold, and this event is to be the arena in which any serious evaluation
of the role of cotton in British recognition of the CSA is to be truly tested. The famine (referred to as such due to the lack of cotton supply, rather than 
any true mass starvation event) has been extensively covered by numerous works, and there exists a large corpus addressing its origins and 
implications for the economy, but the interpretation of these seems to greatly differ across sources. 

An accepted view of the cotton famine may be found in \shortcite{arnoldHistoryCotton1864}, a generally accepted work detailing the event, and which 
presents a picture of a rapidly stagnating goods market in Lancashire\autocite[78]{arnoldHistoryCotton1864}. The all-time highs reached by cotton imports 
in the 1859-60 fiscal years, which were an attempt to correct for the lack of production by the English textile manufacturing industries caused by the 
mild economic depression in the mid 1850s (reflected in the dip in exports in the same period of Figure~\ref{fig: B.i}), were, in many respects, an 
overcompensation. The demand for the goods produced by these mills, mostly coming from the far-eastern markets, was vastly overestimated, and as a result 
the start of the war saw a record surplus of cotton sitting idle in British harbors. 

The great surplus of cotton was quickly turned into a great surplus of cotton-derived goods, which would not be released onto the market until the 
eventual increase in demand in 1863. The period of frantic production and establishment of new mills was, by late 1861, at an end, and this naturally led to
a contraction of the Lancashire operative population (500-600 thousand\autocite[Vol.II, p.13]{adamsBritainAmericanWar1925}). This contraction of the 
operative labor force led to a sharp increase of people in need of social safety, through \flq poor relief\frq\autocite[Vol.II, p.12]{adamsBritainAmericanWar1925},
as well as to a large mass of former industrial workers migrating to non-industrial districts\autocite{arthi2022recessions}. Moreover, an increased rate of
mortality in the districts affected by the cotton famine, as reported by \shortcite{arthi2022recessions}, added more attention to the war, due to the 
(somewhat erroneous) English attribution of much of the blame for the downturn on the Union\autocite[229]{arnoldHistoryCotton1864} \textemdash
If there were any time in which the Parliament were to feel a great enough pressure from the English working-class to prompt their willingness to intervene in the American Civil War, 
it would have been this stage in the cotton famine.

